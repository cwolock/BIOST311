% Create a Table of Contents in Beamer
\documentclass[10pt,t]{beamer}
% Theme choice:
\usetheme{Singapore}
\useoutertheme{sidebar}
\usecolortheme{seahorse}
\setbeamercolor{titlelike}{bg=white}
\setbeamercolor{frametitle}{bg=white}
%\setbeamertemplate{frametitle}[default][left]
\setbeamertemplate{navigation symbols}{}

\usepackage{graphicx}
\usepackage{amsmath}
\usepackage{amsfonts}
\usepackage{amssymb}
\usepackage{amsthm}
\usepackage{ulem}
\usepackage{listings}

% Title page details: 
\title{Chapter 1b: Multiple Linear Regression} 
\author{Taylor Okonek \& Charlie Wolock}
\date{\today}


\begin{document}
	% Title page frame
	\begin{frame}
	\titlepage 
\end{frame}

\begin{frame}{Learning objectives}
By the end of Chapter 1b, you should be able to:
\begin{itemize}
	\item Formulate a regression model, given a scientific or statistical question
	\item Interpret the coefficients for a multiple linear regression model
	\item Interpret confidence intervals and p-values for multiple linear regression coefficients
	\item Classify variables according to their role in a linear regression model (e.g., outcome, predictor, potential confounder, effect modifier, precision variable)
	\item Describe why you would adjust for certain variables in a regression analysis
	\item Use \texttt{R} to fit a multiple linear regression model (and know where in the output to look for the information we need to interpret results)
	\item Create graphs to support your linear regression analysis
\end{itemize}
\end{frame}

% Outline frame
\begin{frame}{Outline}
\tableofcontents
\end{frame}

\AtBeginSection[ ]
{
\begin{frame}{Outline}
\tableofcontents[currentsection]
\end{frame}
}

% Presentation structure


\section{Multiple linear regression: motivation}

\subsection{Linear regression models with multi-level categorical predictors}

\begin{frame}{Linear regression with a categorical predictor}
So far we've seen examples of simple linear regression with a binary predictor. What if instead our scientific question is about the association between a continuous outcome and a \textit{multilevel} (more than two groups) categorical predictor? \pause

\vspace{0.3cm}

If the predictor is binary (e.g. has a genetic variant vs. does not):
\begin{itemize}
	\item Simple linear regression or t-test comparing difference in means between two groups
\end{itemize} \pause

\vspace{0.3cm}

If the predictor is ordinal (e.g. education level: high school, college, masters, etc.):
\begin{itemize}
	\item You \textit{may} be able to find a meaningful way to represent categories numerically (i.e. assign numbers 1 through $k$ to each of $k$ groups)
\end{itemize} \pause

\vspace{0.3cm}

If the predictor is nominal (e.g., country)
\begin{itemize}
	\item No meaningul numeric representation
\end{itemize}

\end{frame}

\begin{frame}{Linear regression with a categorical predictor}
What can we do? \pause We can use \textcolor{blue}{dummy variables}\dots

\vspace{0.3cm}

\textcolor{blue}{Dummy variables}: The set of \textit{binary} variables created by re-writing (or re-coding) a multilevel categorical variable  \pause

\vspace{0.3cm}

Example: Suppose we have a multilevel categorical variable for US region, with values West, Midwest, South, and Northeast

\vspace{0.1cm}

\centering

\includegraphics[scale=0.06]{us_regions.png}

\end{frame}

\begin{frame}{Linear regression with a categorical predictor}
What can we do? We can use \textcolor{blue}{dummy variables}\dots

\vspace{0.3cm}

\textcolor{blue}{Dummy variables}: The set of \textit{binary} variables created by re-writing (or re-coding) a multilevel categorical variable  

\vspace{0.3cm}

Example: Suppose we have a multilevel categorical variable for US region, with values West, Midwest, South, and Northeast

\vspace{0.3cm}

We create the new variables:
\begin{itemize}
	\item Midwest: \textit{Midwest} = 1 if \textit{region} = \textit{Midwest}, and 0 otherwise
	\item South: \textit{South} = 1 if \textit{region} = \textit{South}, and 0 otherwise
	\item Northeast: \textit{Northeast} = 1 if \textit{region} = \textit{Northeast}, and 0 otherwise
\end{itemize} \pause

\vspace{0.1cm}

\textcolor{blue}{Question:} We didn't make a dummy variable for West. Why not? \pause

\textcolor{blue}{Answer:} If all other dummy variables (Midwest, South, and Northeast) are 0, then the region must be West! In this case, West is referred to as a \textcolor{blue}{reference group}.
\end{frame} 

\begin{frame}{Linear regression with a categorical predictor}
Writing out our regression model for this example, we have
$$
E[\text{Outcome} \mid \text{Region}] = \beta_0 + \beta_1 \times \text{Midwest} + \beta_2 \times \text{South} + \beta_3 \times \text{Northeast}
$$

Note that we have more than just an intercept and slope coefficient here (making our way towards multiple linear regression!)

\vspace{0.3cm}

\textcolor{blue}{Question}: How do we interpret the coefficients $\beta_0$, $\beta_1$, $\beta_2$, $\beta_3$? \pause

\vspace{0.3cm}

$\beta_0$: average outcome among those in the Western region

$$
E[\text{Outcome} \mid \text{Region = West}] = \beta_0 + \beta_1 \times 0 + \beta_2 \times 0 + \beta_3 \times 0
$$


\end{frame}

\begin{frame}{Linear regression with a categorical predictor}
Writing out our regression model for this example, we have
$$
E[\text{Outcome} \mid \text{Region}] = \beta_0 + \beta_1 \times \text{Midwest} + \beta_2 \times \text{South} + \beta_3 \times \text{Northeast}
$$

Note that we have more than just an intercept and slope coefficient here (making our way towards multiple linear regression!)

\vspace{0.3cm}

\textcolor{blue}{Question}: How do we interpret the coefficients $\beta_0$, $\beta_1$, $\beta_2$, $\beta_3$?

\vspace{0.3cm}

$\beta_1$: difference in average outcome between groups in the Western region and Midwest region

\begin{align*}
E[&\text{Outcome} \mid  \text{Region = Midwest}] - E[\text{Outcome} \mid \text{Region = West}] \\
& = [\beta_0 + \beta_1 \times 1 + \beta_2 \times 0 + \beta_3 \times 0] - [\beta_0 + \beta_1 \times 0 + \beta_2 \times 0 + \beta_3 \times 0] \\
& = \beta_0 + \beta_1 - \beta_0 \\
& = \beta_1
\end{align*}


\end{frame}

\begin{frame}{Linear regression with a categorical predictor}
Writing out our regression model for this example, we have
$$
E[\text{Outcome} \mid \text{Region}] = \beta_0 + \beta_1 \times \text{Midwest} + \beta_2 \times \text{South} + \beta_3 \times \text{Northeast}
$$

Note that we have more than just an intercept and slope coefficient here (making our way towards multiple linear regression!)

\vspace{0.3cm}

\textcolor{blue}{Question}: How do we interpret the coefficients $\beta_0$, $\beta_1$, $\beta_2$, $\beta_3$?

\vspace{0.3cm}

$\beta_2$: difference in average outcome between groups in the Western region and Southern region

\begin{align*}
E[&\text{Outcome} \mid  \text{Region = South}] - E[\text{Outcome} \mid \text{Region = West}] \\
& = [\beta_0 + \beta_1 \times 0 + \beta_2 \times 1 + \beta_3 \times 0] - [\beta_0 + \beta_1 \times 0 + \beta_2 \times 0 + \beta_3 \times 0] \\
& = \beta_0 + \beta_2 - \beta_0 \\
& = \beta_2
\end{align*}


\end{frame}

\begin{frame}{Linear regression with a categorical predictor}
Writing out our regression model for this example, we have
$$
E[\text{Outcome} \mid \text{Region}] = \beta_0 + \beta_1 \times \text{Midwest} + \beta_2 \times \text{South} + \beta_3 \times \text{Northeast}
$$

Note that we have more than just an intercept and slope coefficient here (making our way towards multiple linear regression!)

\vspace{0.3cm}

\textcolor{blue}{Question}: How do we interpret the coefficients $\beta_0$, $\beta_1$, $\beta_2$, $\beta_3$?

\vspace{0.3cm}

$\beta_3$: difference in average outcome between groups in the Western region and Northeastern region

\begin{align*}
E[&\text{Outcome} \mid  \text{Region = Northeast}] - E[\text{Outcome} \mid \text{Region = West}] \\
& = [\beta_0 + \beta_1 \times 0 + \beta_2 \times 0 + \beta_3 \times 1] - [\beta_0 + \beta_1 \times 0 + \beta_2 \times 0 + \beta_3 \times 0] \\
& = \beta_0 + \beta_3 - \beta_0 \\
& = \beta_3
\end{align*}

\end{frame}

\begin{frame}{Linear regression with a categorical predictor: summary}
Our example:
$$
E[\text{Outcome} \mid  \text{Region = Northeast}] - E[\text{Outcome} \mid \text{Region = West}] 
$$

\begin{itemize}
	\item If we have $k$ categories, we create $k - 1$ \textcolor{blue}{dummy variables}, with the $k$th category being the \textcolor{blue}{reference group} captured in the intercept
	\item \texttt{R} automatically creates these dummy variables for you for multilevel categorical variables, as you'll see on your homework
\end{itemize}

\vspace{0.3cm}

Hypothesis testing: If we want to test whether our outcome is associated with a multilevel categorical variable, we need to test if \textit{all} coefficients for the dummy variables (in this example, $\beta_1, \beta_2, \beta_3$) are equal to zero.

\end{frame}

\begin{frame}{Linear regression with a categorical predictor: Example in \texttt{R}}
Suppose we're interested in whether birthweight is associated with race, in our births dataset. We can fit a linear regression model as before with simple linear regression, and look at the output\dots

\vspace{0.3cm}

\centering

\includegraphics[scale=0.4]{multilevel_cat_lm.png}

\end{frame}

\begin{frame}{Linear regression with a categorical predictor: Example in \texttt{R}}

\begin{figure}
	\centering \includegraphics[scale=0.3]{multilevel_cat_lm.png}
\end{figure}

\vspace{0.1cm}

A couple things to notice:


\end{frame}

\begin{frame}{Linear regression with a categorical predictor: Example in \texttt{R}}

\begin{figure}
	\centering \includegraphics[scale=0.3]{multilevel_cat_lm2.png}
\end{figure}

\vspace{0.1cm}

A couple things to notice:
\begin{itemize}
	\item \texttt{R} has created dummy variables for us! The reference group by default will be the first category alphabetically (in this case, ``asian").
\end{itemize}

\end{frame}

% motivator for multiple linear regression
% note that also had multiple coefficients with the polynomial transformation


\begin{frame}[c]
\centering \huge Any Questions?
\end{frame}

\end{document}