% Create a Table of Contents in Beamer
\documentclass[10pt,t]{beamer}
% Theme choice:
\usetheme{Singapore}
\useoutertheme{sidebar}
\usecolortheme{seahorse}
\setbeamercolor{titlelike}{bg=white}
\setbeamercolor{frametitle}{bg=white}
%\setbeamertemplate{frametitle}[default][left]
\setbeamertemplate{navigation symbols}{}

\usepackage{graphicx}
\usepackage{amsmath}
\usepackage{amsfonts}
\usepackage{amssymb}
\usepackage{amsthm}
\usepackage{ulem}
\usepackage{listings}
\usepackage{xcolor}
\usepackage{wrapfig}
\usepackage{subfig}
\usepackage{setspace}
\usepackage{enumerate}
\usepackage{verbatim}

% new amber color
\definecolor{amber}{rgb}{1.0, 0.75, 0.0}

% Title page details: 
\title{Chapter 3: Survival Analysis} 
\author{Taylor Okonek \& Charlie Wolock}
\date{\today}


\begin{document}
	% Title page frame
\begin{frame}
	\titlepage 
\end{frame}

\begin{frame}{Learning objectives}
	By the end of Chapter 3, you should be able to:
	
	\begin{itemize}
		\item Determine if a variable has been \textcolor{orange}{right-censored}
		\item Discuss the \textcolor{red}{drawbacks} of treating a right-censored variable as binary or continuous
		\item \textbf{Interpret Kaplan-Meier curves}, and create them in \texttt{R}
		\item \textbf{Implement and interpret} a logrank test for equating survival curves
		\item \textbf{Formulate a regression model} given a scientific or statistical question about a right-censored outcome
		\item \textbf{Interpret the coefficients} for a (simple or multiple) proportional hazards regression model
		\item \textbf{Interpret confidence intervals and p-values} for proportional hazards regression coefficients
		\item Use \texttt{R} to fit a proportional hazards regression model and create figures/tables to support your regression analysis
	\end{itemize}
	
\end{frame}

% Outline frame
\begin{frame}{Outline}
	\tableofcontents
\end{frame}

\AtBeginSection[ ]
{
	\begin{frame}{Outline}
		\tableofcontents[currentsection]
	\end{frame}
}


\section{Characteristics of survival data}

\begin{frame}{Quantitative and binary outcomes}
Up to this point, we have focused on questions involving \textcolor{blue}{quantitative} or \textcolor{red}{binary} outcomes: 
\begin{itemize}
\item Is \textcolor{blue}{birthweight} associated with participation in First Steps?
\item Is \textcolor{red}{coronary heart disease} associated with Type A personality? 
\end{itemize}
\end{frame}

\begin{frame}{Time-to-event outcomes}
However, we are often interested in scientific questions that involve \textcolor{green}{time-to-event} outcomes, especially in biomedical settings:
\begin{itemize}
\item Is \textcolor{green}{time to first seizure post operation to remove a brain tumor} associated with pre-operation case review by an epileptologist, in children with both epilepsy and brain tumor?
\item Is \textcolor{green}{time to promotion for university faculty members} associated with gender?
\item Is \textcolor{green}{time to death from any cause} associated with serum levels of C reactive protein?
\end{itemize}
\end{frame}

\begin{frame}
\frametitle{Characteristics of survival data}
{\fontsize{7.5pt}{7.2}\selectfont
Sample of the observed times until first severe panic attack (in weeks):
\begin{tabular}{|c|c|c|c|c|c|c|c|c|c|}
\hline
\textcolor{blue}{meditation} & 14.29 & 7.74 & 21 & 25.18 & 2.83 & 17.57 & 19.13 & 0.14  \\
\hline
placebo & 21 & 3.94 & 7.77 & 6.54 & 24 & 12.55 & 21.29 & 3.58 \\
\hline
\end{tabular}
}

\vspace{0.5cm}
We wish to study whether \textcolor{blue}{meditation} prolongs the \textcolor{green}{time until a severe panic attack} in patients suffering from a panic disorder.

To address this question, we: 
\begin{itemize}
\item recruit 200 patients and randomize them to meditation or placebo;
\item follow each patient until their first severe panic attack post-recruitment;
\item use a t-test to compare the mean time until a severe panic attack in each group
\end{itemize}
\vfill
\begin{footnotesize}
	Note that these data are simulated!
\end{footnotesize}
\end{frame}

\begin{frame}
\frametitle{Characteristics of survival data}
\centering
\includegraphics[width=0.7\textwidth]{figs/meditation_observed_study_time.png}
\end{frame}

\begin{frame}
\frametitle{Characteristics of survival data}
\vspace{-0.8cm}
We're not interested in time from the beginning of the study until a panic attack, but rather the time from when the patient was randomized to meditation/placebo
\begin{center}
\includegraphics[width=0.7\textwidth]{figs/meditation_observed_rand_time.png}
\end{center}
\end{frame}

\begin{frame}
\frametitle{Characteristics of survival data}
In this hypothetical study, we were able to record the actual time of panic attack \textcolor{blue}{for each participant}.
\\ ~\ 

Is this typical of human studies? \pause \textcolor{red}{No!}
\\ ~\ 

Why? \pause Some common reasons are:
\begin{itemize}
\item the study ended (e.g., after 30 weeks) and some participants had not yet had a severe panic attack (\textbf{administrative censoring})
\item the participant left the study before having had a severe panic attack (\textbf{loss to follow-up})
\end{itemize}
If we're watching people and waiting for some health-related event to occur, it's not realistic for us to observe that event for everyone in our study. 
\\ ~\ 

This leads to \textbf{right-censored} data, where rather than knowing the event time exactly, we know that it exceeds some cutoff.
\end{frame}

\begin{frame}
\frametitle{Characteristics of survival data}
\vspace{-0.8cm}
Suppose that we only have resources to run the study for \textbf{30 weeks}, and some people also stop returning to the clinic for follow-up visits. We indicate the last time we saw a patient using a circle. \pause
\begin{center}
\includegraphics[width=0.7\textwidth]{figs/meditation_censored_study_time.png}
\end{center}
\end{frame}

\begin{frame}
\frametitle{Characteristics of survival data}
\centering
\includegraphics[width=0.7\textwidth]{figs/meditation_censored_rand_time.png}
\end{frame}

\begin{frame}
\frametitle{Characteristics of survival data}
{\fontsize{6.5pt}{7.2}\selectfont
The data can be represented as:
\begin{tabular}{|c|c|c|c|c|c|c|c|c|c|}
\hline
\textcolor{blue}{meditation} & 14.29  & \textcolor{red}{7.74+} & \textcolor{red}{21.00+} & 25.18  &  \textcolor{red}{2.83+} & \textcolor{red}{17.57+} & \textcolor{red}{19.13+} &  \textcolor{red}{0.14+}  \\
\hline
placebo & \textcolor{red}{21.00+} &  \textcolor{red}{3.94+} &  \textcolor{red}{7.77+} &  6.54  & \textcolor{red}{24.00+} & 12.55  & 21.29 &  3.58 \\
\hline
\end{tabular}
}
\vspace{0.2cm}
Should we throw out these incompletely observed times?  \pause \textcolor{red}{This is a bad idea!}\pause

\begin{itemize}
\item Censored observations contain valuable information: \pause
{\scriptsize
\begin{itemize}
\item \textcolor{red}{20+}: the participant did not experience a severe panic attack before 20 weeks\pause 
\begin{itemize}
	\item their actual time until the first severe panic attack must be in the interval $(20, +\infty)$\pause
\end{itemize}
\end{itemize}
}
\item Uncensored observations are not representative of the whole study population: valid estimation and inference when excluding missing data (which we have done so far in this course) assumes that the people excluded are ``similar to'' the people still in the data (\textbf{missing completely at random})
\end{itemize}

\end{frame}

\begin{frame}
\frametitle{Missing data}
Think about the following questions:
\begin{enumerate}
\item Are those with smaller or larger times more likely to be censored? Why?\pause 
\item Systematically excluding censored times could lead to \textcolor{red}{biased estimates}. Would this lead to an overestimation or an underestimation of the \textcolor{blue}{mean} time until severe panic attack? Why?
\end{enumerate}

\end{frame}

\begin{frame}
\frametitle{Missing data}
\begin{enumerate}
\item Are those with smaller or larger times more likely to be censored? Why?
\item[] \textcolor{blue}{Those with \textbf{larger times} are more likely to be censored; we only get to observe these people for a comparatively small amount of time, so we may miss their event. For example, people with time from randomization to first panic attack of more than 30 weeks are guaranteed to be censored.}\pause 
\item Systematically excluding censored times could lead to \textcolor{red}{biased estimates}. Would this lead to an overestimation or an underestimation of the mean time until severe panic attack? Why?
\item[] \textcolor{blue}{\textbf{Underestimation}: if we exclude censored times, then we are effectively excluding these people with potentially longer times, and losing all of this information!}
\end{enumerate}

\end{frame}

\begin{frame}
\frametitle{Missing data}
Distribution of times in participants who become censored or uncensored, and the overall target population:\vspace{-0.1cm}
\begin{center}
\includegraphics[height=0.8\textheight]{figs/meditation_density_versus_obs_time.png}
\end{center}
\end{frame}

\begin{frame}
\frametitle{Survival analysis}
\textbf{Survival analysis} is the branch of statistics concerned with the analysis of \textbf{time-to-event data}.

\vspace{0.3cm}

Often, the goals of a survival analysis are to:
\begin{itemize}
\item describe the distribution of a time-to-event variable
\item compare the time-to-event distribution in different subpopulations
\item investigate the relationship between explanatory variables and the time-to-event distribution
\end{itemize}

\vspace{0.3cm}

Even when our time-to-event data do not truly mean survival (e.g., time to first severe panic attack), we still use the term ``survival." 
\end{frame}

\begin{frame}
\frametitle{Characteristics of survival data}
Why makes time-to-event data special? Why not use standard methods?\pause
\begin{itemize}
\item A time-to-event variable is positive and generally skewed\pause 
\item To appropriately define a time-to-event, we must specify:\pause 
{\scriptsize
\begin{itemize}
\item initiating event --- e.g., birth, recruitment into study, onset of disease \pause 
\item terminating event --- e.g., death, onset of disease\pause 
\item time scale --- e.g., calendar time, number of transfusions\pause 
\end{itemize}
}
\item Time-to-event data are generally observed subject to some incompleteness, of which \textbf{censoring} is a major type. Throwing away incomplete data generally results in
{\scriptsize
\begin{enumerate}
\item a loss of information (and increase in estimation uncertainty)
\item biased estimation procedures (most important!)
\end{enumerate}
}
\end{itemize}

\end{frame}

\begin{frame}
\frametitle{Other types of censoring}
A variable is \textbf{censored} if rather than being known exactly, it is known to lie in some set of values.
\\ ~\ 

In this course, we will focus on \textbf{right censoring}, where we only know that the event occurred after the censoring time. This is what we've seen so far. 
\\ ~\ 

Right-censored data are the most common type of censored data found in applications! But you could also have\pause 
\begin{itemize}
	\item Left censoring: We only know the event occurred before the censoring time\pause 
	\item Interval censoring: We only know the event happened between two times, but don't know exactly when
\end{itemize}
\end{frame}

% risk set, plus movie
\begin{frame}
\frametitle{Risk set}
A key concept in survival analysis is the \textbf{risk set}.
\\ ~\ 

\textbf{Risk set at time $t$}: the collection of participants that \textcolor{blue}{could have experienced} the event at time $t$; in other words, those who were still at-risk at time $t$. We denote this $R(t)$. 
\begin{align*}
\textbf{fraction at risk at time } t = & \frac{\textbf{size of risk set at time } t}{\textbf{total sample size}}.
\end{align*}
A few observations:
\begin{itemize}
\item participants \textcolor{blue}{exit} the risk set either when they \textcolor{blue}{experience an event} or when they \textcolor{red}{become right-censored}
\item with right-censored data, all participants are in the risk set at time 0 and the risk set necessarily shrinks over time
\item the \textcolor{blue}{size of the risk set} and the \textcolor{blue}{characteristics of its members} will be critical in survival analysis
\end{itemize}
\end{frame}

\begin{frame}
\frametitle{Risk set}
\begin{center}
\includegraphics[height=0.8\textheight]{figs/risk_set_movie_0.png}
\end{center}
\end{frame}

\begin{frame}
\frametitle{Risk set}
\begin{center}
\includegraphics[height=0.8\textheight]{figs/risk_set_movie_1.png}
\end{center}
\end{frame}

\begin{frame}
\frametitle{Risk set}
\begin{center}
\includegraphics[height=0.8\textheight]{figs/risk_set_movie_2.png}
\end{center}
\end{frame}

\begin{frame}
\frametitle{Risk set}
\begin{center}
\includegraphics[height=0.8\textheight]{figs/risk_set_movie_3.png}
\end{center}
\end{frame}

\begin{frame}
\frametitle{Risk set}
\begin{center}
\includegraphics[height=0.8\textheight]{figs/risk_set_movie_4.png}
\end{center}
\end{frame}

\begin{frame}
\frametitle{Risk set}
\begin{center}
\includegraphics[height=0.8\textheight]{figs/risk_set_movie_5.png}
\end{center}
\end{frame}

\begin{frame}
\frametitle{Risk set}
\begin{center}
\includegraphics[height=0.8\textheight]{figs/risk_set_movie_6.png}
\end{center}
\end{frame}

\begin{frame}
\frametitle{Assumptions}
The assumption of \textbf{independent (or uninformative) censoring} is critical to the vast majority of methods available in survival analysis.
\\ ~\ 

\textcolor{blue}{Independent censoring:} A participant's \textbf{time-to-event} and potential \textbf{censoring time} should be \textbf{independent} of one another.\pause 
\\ ~\ 

What does that mean? \pause \textit{Individuals who are censored at time $t$ must be \textbf{similar} to those remaining in the risk set after time $t$.}\pause 
\\ ~\ 

Independent censoring is a vital assumption, but unfortunately we can't do diagnostics or tests to see if it holds. 

\end{frame}

\begin{frame}{Assumptions}
	When is the independent censoring assumption reasonable? 
	\begin{itemize}
		\item Participants are censored because the study ended? \pause \textcolor{blue}{Yes!}
		\item Participants exited the study because their panic disorder became disabling? \pause \textcolor{red}{No!}
		\item Participants exited the study because their panic disorder dissipated? \pause \textcolor{red}{No!}
		\item Participants exited the study because they moved to a different state? \pause \textcolor{purple}{Maybe?}
	\end{itemize}

\end{frame}

\begin{frame}{Problems with means}
	Fundamentally, a \textcolor{green}{time-to-event} variable is \textcolor{red}{quantitative}, so you might think we'd be interested in the conditional mean.
	\\ ~\ 
	
	Unfortunately, with right censoring, we can't usually estimate the mean.  
	\begin{itemize}
		\item We're missing some data in the right tail of the distribution (i.e., longer survival times)
	\end{itemize}
	\vspace{1cm}
	\textbf{What can we do instead?} 
\end{frame}
% key quantities: density function, survival function, hazard function
\begin{frame}
\frametitle{The survival function}
Suppose that $T$ is a continuous, time-to-event random variable (\textcolor{blue}{$T$ is our outcome of interest in survival analysis}).
\\ ~\ 

The \textbf{survival function}, $S$, is defined as $S(t) := P(T > t)$:\pause 
\begin{itemize}
\item \textcolor{blue}{$S(t) = $ proportion of the population with a time-to-event greater than $t$}\pause 
\item e.g.: if $S(20) = 0.37$, then approximately 37\% of the population will not experience a severe panic attack within the first 20 weeks\pause 
\item $S$ is \textcolor{red}{non-increasing}, starts at 1 [i.e., $S(0) = 1$] and ends at 0 [i.e., $S(\infty) = 0$]
\end{itemize}
\end{frame}

\begin{frame}
\frametitle{The survival function}
\includegraphics[height=0.8\textheight]{figs/survival_function.png}
\end{frame}

\begin{frame}
\frametitle{The hazard function}
The \textbf{hazard function} (or: hazard rate, failure rate) $h$ is
\begin{align*}
h(t) := & \ \lim_{\Delta t \to 0} \frac{P(t \leq T < t + \Delta t \mid T \geq t)}{\Delta t}.
\end{align*}\pause 
\begin{itemize}
\item \textcolor{blue}{$h(t) = $ instantaneous event rate at $t$}\pause 
\item $h(t)\Delta t$ approximates the probability that $T$ is in the interval $[t, t + \Delta t)$ given $T \geq t$\pause 
\item In our panic attack example, $h(t)$ refers to the incidence rate of panic attacks at time $t$ after randomization 
\end{itemize}
The \textbf{cumulative hazard function} $H$ is given by $H(t) := \int_0^t h(u) du$.
\begin{itemize}
	\item Since $h(t)$ is strictly positive, $H(t)$ is an \textcolor{blue}{increasing} function
\end{itemize}

\end{frame}

\begin{frame}{Review: Incidence and prevalence}
	\begin{itemize}
		\item Incidence rate: number of people who develop a condition (incident events) per person per unit time
		\item Cumulative incidence: Fraction of individuals newly acquiring a condition over a specific time period
		\item Prevalence: fraction of individuals with the condition at a specific point in time
	\end{itemize}
	The hazard function $h(t)$ gives the incidence rate at time $t$
\end{frame}


\begin{frame}
\frametitle{Key quantities in survival analysis}
Most methods in survival analysis focus on modeling and/or estimating either the \textcolor{blue}{survival function} or the \textcolor{blue}{hazard rate}.
\\ ~\ 

The hazard rate, while pretty unintuitive, has the desirable property that it can be \textcolor{blue}{easily estimated using right-censored data}.
\begin{itemize}
	\item You can show this using rules of probability. If you're curious we can talk in office hours!
\end{itemize}
\end{frame}

\begin{frame}
\frametitle{Thinking about hazards}
What shape do we expect the hazard function to have? \textcolor{red}{It depends...}
\begin{enumerate}[(a)]
\item survival from surgery until death due to complications
\item survival from onset of a progressive disease
\item the time before a radioactive atom disintegrates
\item an individual's total lifetime
\end{enumerate}
\begin{center}
\includegraphics[width = 0.9\textwidth]{figs/hazard_examples.png}

\underline{\,\phantom{right}\,} \hspace{1.5cm} \underline{\,\phantom{right}\,} \hspace{1.5cm} \underline{\,\phantom{right}\,} \hspace{1.5cm} \underline{\,\phantom{right}\,}
\end{center}
\end{frame}

\begin{frame}
	\frametitle{Thinking about hazards}
	What shape do we expect the hazard function to have? \textcolor{red}{It depends...}
	\begin{enumerate}[(a)]
		\item survival from surgery until death due to complications
		\item survival from onset of a progressive disease
		\item the time before a radioactive atom disintegrates
		\item an individual's total lifetime
	\end{enumerate}
	\begin{center}
		\includegraphics[width = 0.9\textwidth]{figs/hazard_examples.png}
		
		(d) \hspace{1.5cm} (b) \hspace{1.5cm} (a) \hspace{1.5cm} (c)
	\end{center}
\end{frame}

\section{Kaplan-Meier and the log rank test}

\begin{frame}
\frametitle{Reframing survival data}

As we start to think about estimation, it'll be useful to change our notation.
\\ ~\ 

For participant $i$, denote by $T_i$ \textbf{the time to event} and by $C_i$ \textbf{the censoring time}. \pause 
\\ ~\ 

We cannot observe both $T_i$ and $C_i$ -- instead, we observe a pair $(Y_i, \Delta_i)$, where \pause 
\begin{itemize}
\item $Y_i := \min(T_i, C_i)$ is the \textbf{observed follow-up time} \pause
\item $\Delta_i = I(T_i \leq C_i)$ is the \textbf{event indicator} [where $I(X = c)$ is 1 if $X = c$ and zero otherwise] \pause
\end{itemize}

\textbf{Question:} In this new notation, how would we represent the data $\{7.5, 16.6+, 13.5+, 7.4, 14.2\}$?\pause 
\[\{(7.5, 1), (16.6, 0), (13.5, 0), (7.4, 1), (14.2, 1)\}.\]
\end{frame}

\begin{frame}
\frametitle{Nonparametric estimation of a survival function}
Let's start by \textbf{estimating the survival probability at time $t$}: $S(t) = P(T > t)$.
\\ ~\ 

If we observed actual survival times $T_1, T_2, \dots, T_n$, we could take
\begin{align*}
\frac{1}{n}\sum_{i=1}^n I(T_i \geq t)
\end{align*}
as an estimator of $S(t)$ (in other words, just count how many people made it past time $t$ without an event). This is a \textbf{nonparametric} estimator; we do not have to postulate a parametric model for the true survival distribution. This is unlike \textcolor{blue}{classical} linear regression, for example, where we assumed that the errors had a Normal distribution. 
\\ ~\ 

Can we use this estimator in practice? \pause \textcolor{red}{No, since we do not see $T_i$ for participants who are censored!}
\end{frame}

\begin{frame}{Nonparametric estimation of a survival function}
	What about using the observed times?
	\begin{align*}
		\frac{1}{n}\sum_{i=1}^{n}I(Y_i > t)
	\end{align*}
	\textbf{Question:} Would this be an \textcolor{blue}{overestimate} or \textcolor{red}{underestimate} of $P(T > t)$? \pause 
	\\ ~\ 
	
	\textbf{Answer:} It would be an \textcolor{red}{underestimate}. Some people would have survived past time $t$, had they not been censored. (If it helps to think mathematically, we know that $Y_i \leq T_i$ for all $i$, so using $Y_i$ in place of $T_i$ can only make our estimate smaller.)
\end{frame}

\begin{frame}
\frametitle{Nonparametric estimation of a survival function}
\vspace{-0.8cm}
Denote by $u_1 < u_2 < \dots < u_m$ the ordered, distinct, observed times (event or censoring). (\textbf{Question}: When is $m$ equal to $n$, the sample size? \pause \textbf{Answer}: When all the times are unique (no ties).)\pause 
\\ ~\

Also, denote: 
\begin{align*}
d_k = & \ \text{\# of events having occurred at time $u_k$} \\
n_k = & \ \text{\# of individuals \textbf{at risk} at time $u_k$}
\end{align*}
Suppose the data are $\{4, 5, 5+, 8, 12+, 13, 18+, 23, 23, 30\}$.\pause 

\begin{center}
\begin{tabular}{|c|c|c|c|}
\hline
$k$ & time $u_k$ & $d_k$ & $n_k$ \\
\hline
1 & 4 & 1 & 10 \\
2& 5 & 1 & 9 \\
3& 8 & 1 & 7 \\
4&12 & 0 & 6 \\
5&13 & 1 & 5 \\
6&18 & 0 & 4 \\
7&23 & 2 & 3 \\
8&30 & 1 & 1 \\
\hline
\end{tabular}
\end{center}
\end{frame}

\begin{frame}{Conditional and joint probability}
	There's one rule of probability we'll need to use in this section. For events $A$ and $B$
	\begin{itemize}
		\item Let $P(A, B) = P(A \text{ and } B)$ be the joint probability of $A$ and $B$ (probability that both events occur)
		\item Let $P(A \mid B)$ be the conditional probability of $A$ given $B$ (probability that $A$ occurs given that $B$ has occurred)
	\end{itemize}
	It is a fact that
	\begin{align*}
		P(A, B) = P(A \mid B)P(B).
	\end{align*}
\end{frame}

\begin{frame}
\frametitle{Nonparametric estimation of a survival function}
What is a sensible estimate of $P(T > u_1)$?

\begin{columns}
	\begin{column}{0.3\textwidth}
		\begin{center}
\begin{tabular}{|c|c|c|c|}
	\hline
	$k$ & $u_k$ & $d_k$ & $n_k$ \\
	\hline
	1 & 4 & 1 & 10 \\
	2& 5 & 1 & 9 \\
	3& 8 & 1 & 7 \\
	4&12 & 0 & 6 \\
	5&13 & 1 & 5 \\
	6&18 & 0 & 4 \\
	7&23 & 2 & 3 \\
	8&30 & 1 & 1 \\
	\hline
\end{tabular}
	\end{center}
	\end{column}
	\begin{column}{0.7\textwidth}  %%<--- here
		\begin{align*}
			P(T > u_1) &= P(T > u_1, T\geq u_1) \\
			&= P(T > u_1 \mid T \geq u_1)P(T \geq u_1)
		\end{align*}
		How many people made it to \textbf{at least} $u_1$ without an event? \pause \textcolor{blue}{10/10}
		\\ ~\ 
		
		Among those who make it to at least $u_1$ without an event, how many experienced an event \textbf{after} $u_1$? \pause \textcolor{blue}{9/10} 
	\end{column}
\end{columns}
\vspace{0.5cm}
So a reasonable estimate of $P(T > u_1)$ is $\frac{9}{10} \times \frac{10}{10} = 0.9$.
\end{frame}

\begin{frame}
	\frametitle{Nonparametric estimation of a survival function}
	What is a sensible estimate of $P(T > u_2)$?
	
	\begin{columns}
		\begin{column}{0.3\textwidth}
			\begin{center}
\begin{tabular}{|c|c|c|c|}
	\hline
	$k$ & $u_k$ & $d_k$ & $n_k$ \\
	\hline
	1 & 4 & 1 & 10 \\
	2& 5 & 1 & 9 \\
	3& 8 & 1 & 7 \\
	4&12 & 0 & 6 \\
	5&13 & 1 & 5 \\
	6&18 & 0 & 4 \\
	7&23 & 2 & 3 \\
	8&30 & 1 & 1 \\
	\hline
\end{tabular}
			\end{center}
		\end{column}
		\begin{column}{0.7\textwidth}  %%<--- here
			\begin{align*}
				P(T > u_2) &= P(T > u_2, T\geq u_2) \\
				&= P(T > u_2 \mid T \geq u_2)P(T \geq u_2)\\
				&\approx P(T > u_2 \mid T \geq u_2)P(T > u_1)
			\end{align*}
			We already have an estimate of $P(T > u_1)$: \textcolor{blue}{0.9}\pause 
			\\ ~\ 
			
			Among those who make it to at least $u_2$ without an event, how many experienced an event \textbf{after} $u_2$? \pause \textcolor{blue}{8/9}
		\end{column}
	\end{columns}
	\vspace{0.5cm}
	So a reasonable estimate of $P(T > u_2)$ is $\frac{8}{9} \times \frac{9}{10} = 0.8$.
\end{frame}

\begin{frame}
	\frametitle{Nonparametric estimation of a survival function}
	\vspace{-0.5cm}
	What is a sensible estimate of $P(T > u_3)$?
	
	\begin{columns}
		\begin{column}{0.3\textwidth}
			\begin{center}

\begin{tabular}{|c|c|c|c|}
	\hline
	$k$ & $u_k$ & $d_k$ & $n_k$ \\
	\hline
	1 & 4 & 1 & 10 \\
	2& 5 & 1 & 9 \\
	3& 8 & 1 & 7 \\
	4&12 & 0 & 6 \\
	5&13 & 1 & 5 \\
	6&18 & 0 & 4 \\
	7&23 & 2 & 3 \\
	8&30 & 1 & 1 \\
	\hline
\end{tabular}

			\end{center}
		\end{column}
		\begin{column}{0.7\textwidth}  %%<--- here
			\begin{align*}
				P(T > u_3) &= P(T > u_3, T\geq u_3) \\
				&= P(T > u_3 \mid T \geq u_3)P(T \geq u_3)\\
				&\approx P(T > u_3 \mid T \geq u_3)P(T > u_2)
			\end{align*}
			We already have an estimate of $P(T > u_2)$: \textcolor{blue}{0.8}\pause 
			\\ ~\ 
			
			Among those who make it to at least $u_3$ without an event, how many experienced an event \textbf{after} $u_3$? \pause \textcolor{blue}{6/7}
		\end{column}
	\end{columns}
	\vspace{0.5cm}
	So a reasonable estimate of $P(T > u_3)$ is $\frac{6}{7} \times \frac{8}{10} = 0.686$. \textcolor{red}{Note that because of censoring, we don't know if the person censored at $t = 5$ had an event before $t = 8$. But since we assume independent censoring, we treat the people at risk at $t = 8$ as if they are representative of the person censored at $t = 5$.}   
\end{frame}

\begin{frame}
\frametitle{Nonparametric estimation of a survival function}
Continuing in this fashion yields the \textbf{Kaplan-Meier} estimator of $S(t)$: \vspace{-0.3cm}
\begin{align*}
\widehat{S}(t) := \prod_{k: u_k \leq t} \left(1 - \frac{d_k}{n_k} \right) = \prod_{k: u_k \leq t} \left(1 - \frac{\text{\# of events at $u_k$}}{\text{\# at risk at $u_k$}} \right) .
\end{align*}

If any participant is censored at $u_m$ (the final observed time), we consider $\widehat{S}(t)$ to be undefined for $t > u_m$. 

A few observations:
\begin{itemize}
\item $\widehat{S}(t) = 1$ for each $t$ preceding the first event time $u_1$ \pause
\item $\widehat{S}$ is a non-increasing function. Why? \pause
\item $\widehat{S}$ drops to zero at the largest time $u_m$ only if the last observed time is an event (not censoring) \pause
\item the KM estimator only changes at observed event times, and is constant (flat) between observed times \pause
\item the KM estimator relies on \textbf{independent censoring} but does not require postulating a parametric family, and is thus \textbf{nonparametric}
\end{itemize}
\end{frame}

\begin{frame}
\frametitle{Nonparametric estimation of a survival function}

\begin{tabular}{|c|c|c|c|c|c|}
\hline
time $u_k$ & $d_k$ & $n_k$ & $d_k/n_k$ & $1 - d_k/n_k$ & $\widehat{S}(u_k)$ \\
\hline
4 & 1 & 10 & 0.1 & \textcolor{blue}{0.9} & $1 \times \textcolor{blue}{0.9} = \textcolor{blue}{0.9}$\\
5 & 1 & 9 & 0.1111 & \textcolor{blue}{0.8888} & $0.9 \times \textcolor{blue}{0.8888} = \textcolor{blue}{0.8}$ \\
8 & 1 & 7 & 0.1429 & \textcolor{blue}{0.8571} & $0.8 \times \textcolor{blue}{0.8571} = \textcolor{blue}{0.6857}$\\
12 & 0 & 6 & 0 & \textcolor{blue}{1} & $0.6857 \times \textcolor{blue}{1} = \textcolor{blue}{0.6857}$\\
13 & 1 & 5 & 0.2 & \textcolor{blue}{0.8} & $0.6857 \times \textcolor{blue}{0.8} = \textcolor{blue}{0.5486}$\\
18 & 0 & 4 & 0 & \textcolor{blue}{1} & $0.5846 \times \textcolor{blue}{1} = \textcolor{blue}{0.5486}$\\
23 & 2 & 3 & 0.666 & \textcolor{blue}{0.3333} & $0.5486 \times \textcolor{blue}{0.3333} = \textcolor{blue}{0.1829}$\\
30 & 1 & 1 & 1 & \textcolor{blue}{0} & $0.1829 \times \textcolor{blue}{0} = \textcolor{blue}{0}$\\
\hline
\end{tabular}
\end{frame}

\begin{frame}
\frametitle{Nonparametric estimation of a survival function}

\centering
\includegraphics[width=0.7\textwidth]{figs/km_small_example.png}
\end{frame}

\begin{frame}{Another framing of Kaplan-Meier}
	You could also think about Kaplan-Meier as a ``redistribution to the right" estimator, which may be more intuitive. 
	\begin{enumerate}
		\item Arrange the $n$ observed times (events or censorings) in increasing order (if there are ties, censorings go after events)
		\item Assign weight $1/n$ to each observation
		\item Moving from left to right, each time you encounter a censored observation, \textcolor{blue}{redistribute} its weight evenly over all times to its right (events and censoring times)
		\item Approximate the survival probability at each time point by adding up remaining weights at that time point
	\end{enumerate}
\end{frame}

\begin{frame}{Redistribute to the right}
	\begin{itemize}
		\item Redistribute weight at each censoring time
		\item Add up remaining weights at each observed event time
	\end{itemize}
	\begin{footnotesize}
		\begin{tabular}{|c|c|c|c|c|}
			\hline
			$u_k$ & $d_k$ & $n_k$ & Weight per obs. & Remaining weight = $\widehat{S}(u_k)$ \\
			\hline
			4 & 1 & 10 & $\frac{1}{10} = 0.1$ & $9\times 0.1= 0.9$\\
			5 & 1 & 9 & & \\
			8 & 1 & 7 & & \\
			12 & 0 & 6 & & \\
			13 & 1 & 5 & & \\
			18 & 0 & 4 & & \\
			23 & 2 & 3 & & \\
			30 & 1 & 1 & & \\
			\hline
		\end{tabular}
	\end{footnotesize}
	\begin{enumerate}
		\item \textcolor{blue}{no censoring --- no weight redistribution}
		\item \textcolor{blue}{9 subjects remaining after observed event}
	\end{enumerate}
\end{frame}

\begin{frame}{Redistribute to the right}
	\begin{itemize}
		\item Redistribute weight at each censoring time
		\item Add up remaining weights at each observed event time
	\end{itemize}
	\begin{footnotesize}
		\begin{tabular}{|c|c|c|c|c|}
			\hline
			$u_k$ & $d_k$ & $n_k$ & Weight per obs. & Remaining weight = $\widehat{S}(u_k)$ \\
			\hline
			4 & 1 & 10 & $\frac{1}{10} = 0.1$ & $9\times 0.1= 0.9$\\
			5 & 1 & 9 & $0.1$ &  $8\times 0.1= 0.8$  \\
			8 & 1 & 7 & & \\
			12 & 0 & 6 & & \\
			13 & 1 & 5 & & \\
			18 & 0 & 4 & & \\
			23 & 2 & 3 & & \\
			30 & 1 & 1 & & \\
			\hline
		\end{tabular}
	\end{footnotesize}
	\begin{enumerate}
		\item \textcolor{blue}{no censoring --- no weight redistribution}
		\item \textcolor{blue}{8 subjects remaining after observed event}
	\end{enumerate}
\end{frame}

\begin{frame}{Redistribute to the right}
	\begin{itemize}
		\item Redistribute weight at each censoring time
		\item Add up remaining weights at each observed event time
	\end{itemize}
	\begin{footnotesize}
		\begin{tabular}{|c|c|c|c|c|}
			\hline
			$u_k$ & $d_k$ & $n_k$ & Weight per obs. & Remaining weight = $\widehat{S}(u_k)$ \\
			\hline
			4 & 1 & 10 & $\frac{1}{10} = 0.1$ & $9\times 0.1= 0.9$\\
			5 & 1 & 9 & $0.1$ &  $8\times 0.1= 0.8$  \\
			8 & 1 & 7 & $0.1 + (0.1\times \frac{1}{7}) = 0.1143$ & $7 \times 0.1143 = 0.6857$\\
			12 & 0 & 6 & & \\
			13 & 1 & 5 & & \\
			18 & 0 & 4 & & \\
			23 & 2 & 3 & & \\
			30 & 1 & 1 & & \\
			\hline
		\end{tabular}
	\end{footnotesize}
	\begin{enumerate}
		\item \textcolor{blue}{1 censored --- redistribute weight over remaining 7}
		\item \textcolor{blue}{7 subjects remaining after observed event}
	\end{enumerate}
\end{frame}

\begin{frame}{Redistribute to the right}
	\begin{itemize}
		\item Redistribute weight at each censoring time
		\item Add up remaining weights at each observed event time
	\end{itemize}
	\begin{footnotesize}
		\begin{tabular}{|c|c|c|c|c|}
			\hline
			$u_k$ & $d_k$ & $n_k$ & Weight per obs. & Remaining weight = $\widehat{S}(u_k)$ \\
			\hline
			4 & 1 & 10 & $\frac{1}{10} = 0.1$ & $9\times 0.1= 0.9$\\
			5 & 1 & 9 & $0.1$ &  $8\times 0.1= 0.8$  \\
			8 & 1 & 7 & $0.1 + (0.1\times \frac{1}{7}) = 0.1143$ & $7 \times 0.1143 = 0.6857$\\
			12 & 0 & 6 & $0.1143 +  (0.1143\times\frac{1}{5}) = 0.1371$ &  0.6857\\
			13 & 1 & 5 & & \\
			18 & 0 & 4 & & \\
			23 & 2 & 3 & & \\
			30 & 1 & 1 & & \\
			\hline
		\end{tabular}
	\end{footnotesize}
	\begin{enumerate}
		\item \textcolor{blue}{1 censored --- redistribute weight over remaining 5}
		\item \textcolor{blue}{no observed events}
	\end{enumerate}
\end{frame}

\begin{frame}{Redistribute to the right}
	\begin{itemize}
		\item Redistribute weight at each censoring time
		\item Add up remaining weights at each observed event time
	\end{itemize}
	\begin{footnotesize}
		\begin{tabular}{|c|c|c|c|c|}
			\hline
			$u_k$ & $d_k$ & $n_k$ & Weight per obs. & Remaining weight = $\widehat{S}(u_k)$ \\
			\hline
			4 & 1 & 10 & $\frac{1}{10} = 0.1$ & $9\times 0.1= 0.9$\\
			5 & 1 & 9 & $0.1$ &  $8\times 0.1= 0.8$  \\
			8 & 1 & 7 & $0.1 + (0.1\times \frac{1}{7}) = 0.1143$ & $7 \times 0.1143 = 0.6857$\\
			12 & 0 & 6 & $0.1143 +  (0.1143\times\frac{1}{5}) = 0.1371$ &  0.6857\\
			13 & 1 & 5 & 0.1371 & $4 \times 0.1343 =0.5486 $\\
			18 & 0 & 4 & & \\
			23 & 2 & 3 & & \\
			30 & 1 & 1 & & \\
			\hline
		\end{tabular}
	\end{footnotesize}
	\begin{enumerate}
		\item \textcolor{blue}{no censoring --- no weight redistribution}
		\item \textcolor{blue}{4 subjects remaining after observed event}
	\end{enumerate}
\end{frame}

\begin{frame}{Redistribute to the right}
	\begin{itemize}
		\item Redistribute weight at each censoring time
		\item Add up remaining weights at each observed event time
	\end{itemize}
	\begin{footnotesize}
		\begin{tabular}{|c|c|c|c|c|}
			\hline
			$u_k$ & $d_k$ & $n_k$ & Weight per obs. & Remaining weight = $\widehat{S}(u_k)$ \\
			\hline
			4 & 1 & 10 & $\frac{1}{10} = 0.1$ & $9\times 0.1= 0.9$\\
			5 & 1 & 9 & $0.1$ &  $8\times 0.1= 0.8$  \\
			8 & 1 & 7 & $0.1 + (0.1\times \frac{1}{7}) = 0.1143$ & $7 \times 0.1143 = 0.6857$\\
			12 & 0 & 6 & $0.1143 +  (0.1143\times\frac{1}{5}) = 0.1371$ &  0.6857\\
			13 & 1 & 5 & 0.1371 & $4 \times 0.1343 =0.5486 $\\
			18 & 0 & 4 & $0.1371 + (0.1371\times \frac{1}{3}) = 0.1829$ & 0.5486\\
			23 & 2 & 3 & & \\
			30 & 1 & 1 & & \\
			\hline
		\end{tabular}
	\end{footnotesize}
	\begin{enumerate}
		\item \textcolor{blue}{1 censored --- redistribute weight over remaining 3}
		\item \textcolor{blue}{no observed events}
	\end{enumerate}
\end{frame}

\begin{frame}{Redistribute to the right}
	\begin{itemize}
		\item Redistribute weight at each censoring time
		\item Add up remaining weights at each observed event time
	\end{itemize}
	\begin{footnotesize}
		\begin{tabular}{|c|c|c|c|c|}
			\hline
			$u_k$ & $d_k$ & $n_k$ & Weight per obs. & Remaining weight = $\widehat{S}(u_k)$ \\
			\hline
			4 & 1 & 10 & $\frac{1}{10} = 0.1$ & $9\times 0.1= 0.9$\\
			5 & 1 & 9 & $0.1$ &  $8\times 0.1= 0.8$  \\
			8 & 1 & 7 & $0.1 + (0.1\times \frac{1}{7}) = 0.1143$ & $7 \times 0.1143 = 0.6857$\\
			12 & 0 & 6 & $0.1143 +  (0.1143\times\frac{1}{5}) = 0.1371$ &  0.6857\\
			13 & 1 & 5 & 0.1371 & $4 \times 0.1343 =0.5486 $\\
			18 & 0 & 4 & $0.1371 + (0.1371\times \frac{1}{3}) = 0.1829$ & 0.5486\\
			23 & 2 & 3 & 0.1829&  $1 \times 0.1829 = 0.1829$\\
			30 & 1 & 1 & & \\
			\hline
		\end{tabular}
	\end{footnotesize}
	\begin{enumerate}
		\item \textcolor{blue}{no censoring --- no weight redistribution}
		\item \textcolor{blue}{1 subject remaining after 2 observed events}
	\end{enumerate}
\end{frame}

\begin{frame}{Redistribute to the right}
\begin{itemize}
	\item Redistribute weight at each censoring time
	\item Add up remaining weights at each observed event time
\end{itemize}
		\begin{footnotesize}
		\begin{tabular}{|c|c|c|c|c|}
			\hline
				$u_k$ & $d_k$ & $n_k$ & Weight per obs. & Remaining weight = $\widehat{S}(u_k)$ \\
			\hline
			4 & 1 & 10 & $\frac{1}{10} = 0.1$ & $9\times 0.1= 0.9$\\
			5 & 1 & 9 & $0.1$ &  $8\times 0.1= 0.8$  \\
			8 & 1 & 7 & $0.1 + (0.1\times \frac{1}{7}) = 0.1143$ & $7 \times 0.1143 = 0.6857$\\
			12 & 0 & 6 & $0.1143 +  (0.1143\times\frac{1}{5}) = 0.1371$ &  0.6857\\
			13 & 1 & 5 & 0.1371 & $4 \times 0.1343 =0.5486 $\\
			18 & 0 & 4 & $0.1371 + (0.1371\times \frac{1}{3}) = 0.1829$ & 0.5486\\
			23 & 2 & 3 & 0.1829&  $1 \times 0.1829 = 0.1829$\\
			30 & 1 & 1 & 0.1829 & $0.1829 \times 0 = 0$\\
			\hline
		\end{tabular}
	\end{footnotesize}
\end{frame}

\begin{frame}
\frametitle{Nonparametric estimation of a survival function}

Our estimate of $\widehat{S}(t)$ has a corresponding standard error. 
\begin{align*}
\widehat{SE}(t) := \widehat{S}(t)\sqrt{\sum_{i : u_i \leq t}\frac{d_i}{n_i(n_i - d_i)}}.
\end{align*}
This suggests using 
\textcolor{blue}{[$\widehat{S}(t) - 1.96 \widehat{SE}(t)$, $\widehat{S}(t) + 1.96 \widehat{SE}(t)$]} as a 95\% confidence interval for $S(t)$.
\\ ~\ 

As before, \texttt{R} can calculate these for you.
\end{frame}

\begin{frame}
\frametitle{Nonparametric estimation of a survival function}
\begin{tabular}{|c|c|c|c|c|c|c|}
\hline
time $u_k$ & $d_k$ & $n_k$ & $\widehat{S}(u_k)$ & $\widehat{SE}(t)$ & CI lower & CI upper \\
\hline
4 & 1 & 10 & 0.9 & 0.0949 & \textcolor{blue}{0.7141} & \textcolor{red}{1.0859} \\
5 & 1 & 9 &  0.8 & 0.0949 & \textcolor{blue}{0.5521} & \textcolor{red}{1.0479}\\
8 & 1 & 7 & 0.6857 & 0.0949 & \textcolor{blue}{0.3888} & \textcolor{blue}{0.9826}\\
12 & 0 & 6 & 0.6857 & 0.0949 & \textcolor{blue}{0.3888} & \textcolor{blue}{0.9826}\\
13 & 1 & 5 & 0.5486 & 0.0949 & \textcolor{blue}{0.2106} & \textcolor{blue}{0.8866}\\
18 & 0 & 4 & 0.5486 & 0.0949 & \textcolor{blue}{0.2106} & \textcolor{blue}{0.8866}\\
23 & 2 & 3 & 0.1829 & 0.0949 & \textcolor{red}{-0.1307} & \textcolor{blue}{0.4965}\\
30 & 1 & 1 & 0 & & & \\
\hline
\end{tabular}

\vspace{0.5cm}
(Other methods will ensure that your intervals stay inside $[0,1]$, but the details are beyond the scope of our class. Be careful when interpreting.)
\end{frame}

\begin{frame}{\texttt{mayo} dataset}
	Let's dig into some real data!
	\\ ~\ 
	
	The \texttt{mayo} dataset comes from a \textcolor{blue}{randomized controlled trial} of d-penicillamine for treatment of primary biliary cirrhosis (PBC). PBC is a liver disease that affects the bile ducts, and can eventually lead to severe liver damage, called cirrhosis. There's some evidence that PBC has an autoimmune component, and d-penicillamine was considered due to its possible effect on autoimmune processes. 
	\\ ~\ 
	
	In this trial, 311 patients were randomized to receive either treatment or placebo. Baseline health information was collected (mostly about baseline liver function), and patients were followed until death or censoring (due to either administrative censoring or dropout). 
\end{frame}

\begin{frame}{\texttt{mayo} dataset}
	A subset of relevant variables in this dataset:
	\begin{itemize}
		\item \texttt{obstime}: Time from randomization until death or last follow-up
		\item \texttt{status}: Indicator variable of whether death was observed
		\item \texttt{treatment}: Indicator of treatment group
		\item \texttt{age}: Age at baseline (years)
		\item \texttt{sex}: Binarized sex (1 = female, 0 = male)
		\item \texttt{edema}: Indicator of edema (swelling in legs indicative of decreased liver function)
		\item \texttt{stage}: Disease stage at baseline (1-4 in increasing order of severity)
		\item \texttt{bili}: level of bilirubin in blood, indicator of liver function (higher is worse) 
	\end{itemize}
	We'll leave the descriptives to you!
\end{frame}

\begin{frame}{Nonparametric estimation of a survival function}
	Let's start by taking a look at the Kaplan-Meier survival curves in the two treatment arms. Most of the survival analysis functions we'll use in Chapter 3 involve the \texttt{survival} and \texttt{survminer} packages. 
	\begin{center}
		\includegraphics[width=0.8\textwidth]{figs/KM_unstrat_code.png}
	\end{center}
\end{frame}

\begin{frame}
\frametitle{Nonparametric estimation of a survival function} 
\begin{center}
\includegraphics[width=\textwidth]{figs/KM_unstrat.png}
\end{center}
\end{frame}

\begin{frame}
	\frametitle{Nonparametric estimation of a survival function} 
	We stratify by treatment group using the formula syntax in \texttt{survfit}:
	\begin{center}
		\includegraphics[width=\textwidth]{figs/KM_strat_code.png}
	\end{center}
\end{frame}

\begin{frame}
	\frametitle{Nonparametric estimation of a survival function} 
	\begin{center}
		\includegraphics[width=\textwidth]{figs/KM_strat.png}
	\end{center}
	What's missing from this plot? \pause Confidence bands!
\end{frame}

\begin{frame}
	\frametitle{Nonparametric estimation of a survival function} 
	\begin{center}
			\includegraphics[height=0.3\textheight]{figs/KM_strat_CI_code.png}
		\includegraphics[width=\textwidth]{figs/KM_strat_CI.png}
	\end{center}
\end{frame}

\begin{frame}
\frametitle{Nonparametric estimation of measures of central tendency}

With continuous or binary data, we often used the mean or median to describe the distribution of the variable.
\\ ~\ 

Can we describe the \textbf{central tendency of a distribution} using right-censored data?
\\ ~\ 

Should we (or can we) use the \textbf{mean} or the \textbf{median}?
\begin{itemize}
\item The mean may be simpler to communicate.
\item The median is robust to outliers, while the mean is not.
\item Recall that in survival analysis, the mean can be problematic because \textcolor{red}{censoring may render the right tail of the distribution ``unidentifable."}
\end{itemize}
\end{frame}

\begin{frame}
\frametitle{Nonparametric estimation of measures of central tendency}

For this reason, people tend to focus on the \textbf{restricted mean}. We pick some time $\tau$ and integrate the survival curve up to $\tau$: $\mu_\tau := \int_0^\tau S(u) du$; an estimator is given by $\widehat{\mu}_\tau := \int_0^\tau \widehat{S}(u)du$.
\\ ~\ 

If $\tau$ is not too large, this may be well-defined. 
\\ ~\ 

The restricted mean is interpreted as the \textcolor{blue}{average survival time, restricted to times less than $\tau$}, or the average of $\min\{T, \tau\}$.
\\ ~\ 

\textcolor{red}{However, the interpretation of $\mu_\tau$ may not be of interest}, leading people to consider the median instead.
\end{frame}

\begin{frame}
\frametitle{Nonparametric estimation of measures of central tendency}
Whenever $S$ is continuous and strictly decreasing, we define the median as the \textbf{timepoint $t_{0.5}$ such that $S(t_{0.5}) = 1/2$}.

\begin{center}
\includegraphics[width=0.7\textheight]{figs/survival_function_median.png}
\end{center}
\end{frame}

\begin{frame}
\frametitle{Nonparametric estimation of measures of central tendency}
In practice, we will use as estimator of $t_{0.5}$ the median $\hat{t}_{0.5}$ of $\widehat{S}$. But the survival function $\widehat{S}$ is \textbf{not continuous nor strictly decreasing}!

\begin{center}
\includegraphics[width=0.6\textheight]{figs/km_small_example_median.png}
\end{center}
Here, $\hat{S}(t)$ jumps from above 0.5 to below 0.5 without hitting it exactly. We'll need to think about what we do here. 
\end{frame}
 

\begin{frame}
\frametitle{Nonparametric estimation of measures of central tendency}
Some other potential issues: 
\begin{enumerate}
\item What if there is no point $\hat{t}_{0.5}$ at which $\widehat{S}$ goes from above to below 0.5?
\begin{center}
	\includegraphics[width=0.6\textheight]{figs/no_median.png}
\end{center}
\item What happens if $\widehat{S}$ equals 0.5 over a whole interval?
\end{enumerate}

\end{frame}


\begin{frame}{Nonparametric esitmation of measures of central tendency}
	To be precise, we'll define the mean as the \textcolor{blue}{smallest value $t$ such that $\widehat{S}(t) \leq 0.5$}. This covers us in case the survival curve ``jumps over" 0.5 without hitting it, or is equal to 0.5 over an entire interval of times. 
	\\ ~\ 
	
	In addition, if the estimated survival curve never crosses 0.5, we simply cannot estimate the median survival time. 
\end{frame}


\begin{frame}
	\vspace{-0.5cm}
	Adding the median to a KM curve:
	\frametitle{Nonparametric estimations of measures of central tendency} 
	\begin{center}
		\includegraphics[height=0.3\textheight]{figs/KM_strat_medians_code.png}
		\includegraphics[width=\textwidth]{figs/KM_strat_medians.png}
	\end{center}
\end{frame}

\begin{frame}
\frametitle{Nonparametric testing of equal survivorship between two groups}

It is often of interest to \textbf{contrast the survivorship of several groups}, rather than simply describing the survival time distribution in a single group. For example:
\begin{itemize}
\item Does time to death or liver transplantation depend on treatment with d-penicillamine? 
\end{itemize}

Commonly, in survival analysis, we \textcolor{blue}{compare the entire survival function between two groups} to make this contrast.
\end{frame}

\begin{frame}
\frametitle{Nonparametric testing of equal survivorship between two groups}

Comparing the entire survival function is the same as saying that the group-specific time-to-event distributions are identical. Suppose we have groups $Z = 0$ and $Z = 1$, and our outcome is death.
\\ ~\ 

\textbf{Question:} If we want to test for identical time-to-event distributions, what should our null and alternative hypotheses be? \pause 
\\ ~\ 

\textbf{Answer:}
\begin{align*}
\textcolor{blue}{H_0: S_0(t) = S_1(t) \text{ for all } t} \ \text{vs} \ \textcolor{orange}{H_a: S_0(t) \neq S_1(t) \text{ for some }t}.
\end{align*} \pause 
Note that we need to compare survival across all times! How do we do this? 
\end{frame}

\begin{frame}{Nonparametric testing of equal survivorship between two groups}
At any observed failure time $t$, we can construct a $2 \times 2$ table:\vspace{-0.2cm}

\begin{center}
	\begin{tabular}{c|c|c|c}
		& $Z = 0$ & $Z = 1$ \\
		\hline
		died & $d_0(t)$ & $d_1(t)$ & $d(t)$ \\
		did not die & $n_0(t) - d_0(t)$ & $n_1(t) - d_1(t)$ & $n(t) - d(t)$\\
		at risk & $n_0(t)$ & $n_1(t)$ & $n(t)$
	\end{tabular}
\end{center}\vspace{-0.2cm}

If $H_0$ is true, then we expect that the rows and columns are independent. That means we can compute the \textcolor{blue}{expected counts} of deaths and compare to the observed counts. TAYLOR: I don't think it's clear why you expect rows and columns to be independent based on our null hypothesis. I also don't think it's clear why you're comparing expecteds to observeds to see if the survival curves in each group are equal
\end{frame}

\begin{frame}
\frametitle{Nonparametric testing of equal survivorship between two groups}

\begin{center}
\begin{tabular}{c|c|c|c}
& $Z = 0$ & $Z = 1$ \\
\hline
died & $d_0(t)$ & $d_1(t)$ & $d(t)$ \\
did not die & $n_0(t) - d_0(t)$ & $n_1(t) - d_1(t)$ & $n(t) - d(t)$\\
at risk & $n_0(t)$ & $n_1(t)$ & $n(t)$
\end{tabular}
\end{center}\vspace{-0.2cm}

We'll only focus on the top-left cell (although we could just as easily choose any of them).
\\ ~\ 

What do we expect to see in the top-left cell under $H_0$?
\begin{align*}
\textbf{observed count } o(t) &= d_0(t) \\
\textbf{expected count } e(t) &= n(t) \times P(Z = 0) \times P(\text{died}) \text{  (\textcolor{red}{independence!})}\\
&\approx n(t)\times \frac{n_0(t)}{n(t)}\times  \frac{d(t)}{n(t)} = \frac{d(t)n_0(t)}{n(t)}
\end{align*}
\textcolor{blue}{The difference $o(t) - e(t)$ should be small under $H_0$.} 
\end{frame}

\begin{frame}
\frametitle{Nonparametric testing of equal survivorship between two groups}
Conditional on the margins, and under $H_0$, this difference has variance equal to $v(t) := \frac{n_0(t)n_1(t)d(t)\{n(t)-d(t)\}}{n(t)^2\{n(t) - 1\}}$.
\\ ~\ 

We can aggregate $o(t) - e(t)$ over \textcolor{orange}{all distinct observed failure times} to get the \textbf{log-rank test statistic}
\begin{align*}
\widehat{T}^2_{LR, n} := \left[\frac{\sum_i \{o(t_i) - e(t_i)\}}{\sqrt{\sum_iv(t_i)}} \right]^2,
\end{align*}

\textcolor{cyan}{Under $H_0$, with large sample size the distribution of $\widehat{T}^2_{LR, n}$ is approximately $\chi^2_1$}: thus, if we observe $\widehat{T}^2_{LR, n} = x$, then the p-value is given by $P(\chi^2_1 > x)$!
\\ ~\ 

\textbf{You will never need to do this by hand!}
\end{frame}

\begin{frame}
\frametitle{Nonparametric testing of equal survivorship between two groups}

\begin{center}
	\includegraphics[width=\textwidth]{figs/logrank_code.png}
\end{center}

Using a log-rank test of the null hypothesis of no survival difference at any time between treatment and placebo against the alternative of a survival difference, we fail to reject the null hypothesis at a 0.05 significance level ($p = 0.8$). We do not find statistically significant evidence of a survival difference. 

\end{frame}

\begin{frame}[fragile]
\frametitle{Motivation for regression modeling}
The log-rank test is nice for testing for differences between two groups. 
\\ ~\ 

If we have a \textcolor{red}{confounder}, we can even adapt the log-rank test by breaking the confounder into discrete categories. But this is pretty limited when we have multiple confounders. 
\\ ~\

In addition, we'd like to get some \textcolor{blue}{estimate} of association, rather than just a hypothesis test. 
\\ ~\ 

This is where regression modeling comes in!
\end{frame}

\section{Semiparametric estimation and inference}
\begin{frame}
\frametitle{Motivation for Cox proportional hazards regression}
\vspace{-0.8cm}
With a continuous predictor, we could split the predictor into two subgroups and perform a nonparametric analysis to compare the survivorship between the subgroups.
\\ ~\ 

For example, we could split \texttt{age} into those under 50 years old, and those over 50 years old, and compare survival:

\begin{center}
\includegraphics[width = \textwidth]{figs/KM_strat_age.png}
\end{center}
\end{frame}

\begin{frame}{Motivation for Cox proportional hazards regression}
	A log rank test tests the null hypothesis of no survival difference between under and over 50 age groups:
	\begin{center}
		\includegraphics[width = \textwidth]{figs/logrank_agecat.png}
	\end{center}
\end{frame}

\begin{frame}
\frametitle{Motivation for Cox proportional hazards regression}

It appears that the survival probabilities are generally lower (at any point in time) for older versus younger patients. 
\\ ~\ 

\textbf{Question:} What does this suggest about the \textcolor{blue}{hazards} in the two age groups? \pause

\textbf{Answer:} Because the younger age group has higher survival probability, we can infer that it also has a lower hazard. 
\\ ~\ 

If we want to relate the hazard functions between two groups, we might consider a model like
\begin{align*}
\textcolor{orange}{h_1(t)} &= c\textcolor{cyan}{h_0(t)},
\end{align*}
where $\textcolor{orange}{h_1(t)}$ and $\textcolor{cyan}{h_0(t)}$ are the subgroup-specific hazard functions.
\end{frame}

\begin{frame}
\frametitle{The hazard ratio}
If we only have two groups to compare, we might think about imposing a model on the \textcolor{blue}{ratio of the hazards} of the two groups:
\begin{align*}
\frac{\textcolor{orange}{h_1(t)}}{\textcolor{cyan}{h_0(t)}} &= c
\end{align*}
This is the simplest example of a \textbf{proportional hazards} model, in which \textit{the hazard functions corresponding to every subgroup considered are proportional to one another.}
\\ ~\ 

To be a bit more precise, we can re-express this model as
\begin{align*}
\frac{\textcolor{orange}{h_1(t)}}{\textcolor{cyan}{h_0(t)}} &= c \text{ for every time } t,
\end{align*}
and so, the \textbf{hazard ratio comparing the two subgroups} is \textcolor{red}{constant over time}.

\end{frame}

\begin{frame}
\frametitle{The hazard ratio}
The hazard ratio at time $t$ tells us how much more likely it is that an event will happen in subgroup 1 at time $t$ versus in subgroup 0 at time $t$ \textit{given that it has not yet happened in either subgroup.}
\begin{align*}
\frac{\textcolor{orange}{h_1(t)}}{\textcolor{cyan}{h_0(t)}} = \frac{\textcolor{orange}{h_1(t)}\Delta t}{\textcolor{cyan}{h_0(t)}\Delta t} \approx \frac{\textcolor{orange}{P(t \leq T < t + \Delta t \mid T \geq t, \text{subgroup }1)}}{\textcolor{cyan}{P(t \leq T < t + \Delta t \mid T \geq t, \text{subgroup }0)}}.
\end{align*}

\end{frame}

\begin{frame}
\frametitle{The hazard ratio}
\begin{align*}
	\frac{\textcolor{orange}{h_1(t)}}{\textcolor{cyan}{h_0(t)}} &= c \text{ for every time } t,
\end{align*}
What does $c < 1$ imply? \pause A \textbf{decreased hazard} in subgroup 1 relative to subgroup 0. Subgroup 1 is at lower instantaneous risk than subgroup 0 by a factor of $c$.\pause 
\\ ~\ 

What does $c > 1$ imply? \pause An \textbf{increased hazard} in subgroup 1 relative to subgroup 0.  Subgroup 1 is at higher instantaneous risk than subgroup 0 by a factor of $c$.
\end{frame}

\begin{frame}{The hazard ratio}
	\textbf{Question:} Based on our Kaplan-Meier curves for survival by age group, what can we say about the sign of the following hazard ratio?
	\begin{align*}
		\frac{h_{\text{over 50}}(t)}{h_{\text{under 50}}(t)}
	\end{align*} \pause 
	\textbf{Answer:} Since we saw lower survival over all times for the older versus the younger group (i.e., a larger hazard), we'd expect this hazard ratio to be greater than 1.
\end{frame}

\begin{frame}
\frametitle{The hazard ratio}
Assume that the subgroups are defined so that subgroup 1 has some exposure that we want to show is beneficial (e.g. treatment), while subgroup 0 does not (e.g. placebo):

\begin{enumerate}
\item If the time-to-event is, e.g., transplant $\rightarrow$ organ rejection or onset of disease $\rightarrow$ death, do we want to observe $c < 1$ or $c > 1$? Why?
\item If the time-to-event is, e.g., treatment $\rightarrow$ remission, do we want to observe $c < 1$ or $c > 1$? Why?
\end{enumerate}
\end{frame}

%\begin{comment}
\begin{frame}
\frametitle{The hazard ratio}
\begin{enumerate}
\item If the time-to-event is, e.g., transplant $\rightarrow$ organ rejection or onset of disease $\rightarrow$ death, do we want to observe $c < 1$ or $c > 1$? Why?
\item[] \textcolor{blue}{$c < 1$, because this means that the event tends to occur \textbf{less often} in subgroup 1 than subgroup 0, and events are bad!}
\item If the time-to-event is, e.g., treatment $\rightarrow$ remission, do we want to observe $c < 1$ or $c > 1$? Why?
\item[] \textcolor{blue}{$c > 1$, because this means that the event tends to occur \textbf{more often} in subgroup 1 than subgroup 0, and events are good!}
\end{enumerate}
\end{frame}
%\end{comment}

\begin{frame}
\frametitle{Cox proportional hazards regression}
In practice, we often have more than two subgroups. In fact, our predictor might be continuous, so defining subgroups isn't straightforward.
\\ ~\ 

Denoting the hazard function of $T$ given a single predictor of interest $X$ at time $t$ by $h(t \mid X)$, the \textbf{simple proportional hazards model} is:
\begin{align*}
\log h(t \mid X) = & \ \log h_0(t) + \beta_1 X,
\end{align*}
for each $t$, where $h_0$ is an \textcolor{blue}{unspecified\textsuperscript{*}} hazard function. The model above is the basis of what we call the \textbf{Cox proportional hazards model.}
\\ ~\ 

How does this relate to the hazard ratio? We'll see!
\vfill
\small{\textsuperscript{*}What does this mean? It turns out that to estimate the $\beta_1$ parameter of this model, we don't ever have to deal with $h_0$.}
\end{frame}

\begin{frame}
\frametitle{Cox proportional hazards regression}
Exponentiating both sides of the previous equation, we can model the hazard function as
\begin{align*}
h(t \mid X) = & \ h_0(t)e^{\beta_1 X}.
\end{align*}

Since a survival distribution is completely determined by its hazard function, the distribution of $T$ given $X$ is composed of
\begin{itemize}
\item a \textcolor{blue}{coefficient of interest} $\beta_1$: this is how we relate $X$ to the hazard
\item a \textcolor{blue}{baseline hazard function} $h_0$ (often not of interest): doesn't involve $X$ at all
\end{itemize}
\end{frame}

\begin{frame}
\frametitle{Cox proportional hazards regression: interpretation}
\vspace{-0.5cm}
To interpret $\beta_1$, we use our standard approach: 
\begin{align*}
h(t \mid X = x + 1) = & \ h_0(t)e^{\beta_1 (x + 1)}\\
h(t \mid X = x) = & \ h_0(t)e^{\beta_1 x} 
\end{align*}\pause \vspace{-0.3cm}
\begin{align*}
\textcolor{blue}{\frac{h(t \mid X = x + 1)}{h(t \mid X = x)}} = & \ \frac{h_0(t)e^{\beta_1 (x + 1)}}{h_0(t)e^{\beta_1}} \\
=& \ \frac{e^{\beta_1x + \beta_1}}{e^{\beta_1 x}} \\
=& \ \textcolor{blue}{e^{\beta_1 }}
\end{align*}
So $e^{\beta_1}$ is the \textbf{hazard ratio} comparing two groups at the same time that differ in $X$ by one unit. 
\\ ~\ 

We could also say it's the  \textbf{ratio of the instantaneous risk of the event} comparing two groups at the same time that differ in $X$ by one unit.
\end{frame}

\begin{frame}{Cox proportional hazards regression in \texttt{R}}
	Estimates of the $\beta$ parameters from proportional hazards regression are (approximately) normally distributed for large sample size, like all of our other regression estimators.
	\\ ~\ 

	The syntax for Cox proportional hazards regression in \texttt{R} is similar to fitting a linear model or GLM, except
	\begin{enumerate}
		\item We use the function \texttt{coxph} in the \texttt{survival} package in \texttt{R}
		\item Our outcome is a \texttt{Surv} object
	\end{enumerate}
	\begin{center}
		\texttt{coxph(Surv(time, event) $\sim$ X1 + \dots)}
	\end{center}
	\begin{itemize}
	\item This provides naive standard errors. Robust implementations do exist.
	\item Note that \texttt{coxph}, unlike \texttt{glm}, gives us exponentiated coefficients and confidence intervals! 
\end{itemize}
\end{frame}

\begin{frame}{Simple Cox regression: example}
	Rather than having to arbitrarily binarize age and do a log-rank test, let's instead fit a Cox model: 
	\begin{center}
		\includegraphics[width = \textwidth]{figs/cox_age.png}
	\end{center}
	Use the first p-value in the display (7.41 $\times 10^{-6}$).
\end{frame}

\begin{frame}{Simple Cox regression: example}
	We got a hazard ratio of 1.04, with 95\% CI (1.023 - 1.059). In context:
	\\ ~\ 
	
	\textcolor{blue}{We estimate the hazard ratio of death comparing groups one year apart in age is 1.04, with the older group having the higher hazard. The data would not be surprising if the true hazard ratio were between 1.023 and 1.059. We reject the null hypothesis of a hazard ratio of 1 against the alternative that the hazard ratio is not 1 (p $<$ 0.001). There is significant evidence of an association between age and the instantaneous risk of death.}
\end{frame}

\begin{frame}
\frametitle{Multiple Cox proportional hazards regression}
With more than one covariate, we simply extend the model:
\begin{align*}
\log \{h(t \mid X_1, \dots, X_p)\} = & \ \log\{h_0(t)\} + \beta_1 X_1 + \dots + \beta_pX_p \\
h(t \mid X_1, \dots, X_p) = & \ h_0(t)e^{\beta_1X_1 + \dots + \beta_pX_p}.
\end{align*}
\pause 
Now, we interpret the regression parameter $e^{\beta_1}$ as: \pause \textcolor{blue}{the hazard ratio comparing two groups that differ in $X_1$ by one unit, but are the same with respect to $X_2, \dots, X_p$.} \pause 
\\ ~\ 

We interpret $e^{\beta_j}$, for $j = 2, \dots, p$, as: \pause
\textcolor{blue}{the hazard ratio comparing two groups that differ in $X_j$ by one unit, but are the same with respect to $X_1, \dots, X_{j-1}, X_{j+1}, \dots, X_p$.} \pause 
\\ ~\

Always keep in mind that hazard = ``instantaneous risk of an event." The term ``hazard ratio" by itself isn't very informative, but it's the more common way to phrase this.
\end{frame}

\begin{frame}{Cox proportional hazards regression: adjustment variables}
	Adjusting for \textcolor{red}{potential confounders}, as in all other types of regression we have seen, is necessary if we have observational data and wish to estimate an association between two variables. We do this by adding the potential confounders as covariates in our model.
	\\ ~\ 
	
	When adjusting for \textcolor{green}{precision variables}, as in logistic regression, we don't necessarily reduce our standard errors, so whether to include them or not depends on our scientific question. 
	\\ ~\ 
	
	Finally, as in linear and logistic regression, \textcolor{blue}{effect modification} is assessed using an interaction term that is interpreted as a ratio of ratios.
\end{frame}

\begin{frame}{Multiple Cox regression: example}
	\vspace{-0.6cm}
	Another variable in our dataset is \texttt{stage}, which is an ordinal categorical variable (4 categories) describing the stage of disease at baseline. Here are boxplots of \texttt{age} by \texttt{stage}: 
	\begin{center}
		\includegraphics[width = \textwidth]{figs/stage_by_age_box.png}
	\end{center}
	If we have reason to believe \texttt{stage} may have a causal effect on survival time, what role does \texttt{stage} play in the association between \texttt{age} and survival?
\end{frame}

\begin{frame}{Multiple Cox regression: example}
	So, in order to deal with possible confounding by \texttt{stage}, let's include it in our model:
		\begin{center}
		\includegraphics[width = \textwidth]{figs/multiple_cox_regression_code.png}
	\end{center}
\end{frame}

\begin{frame}{Multiple Cox regression: example}
		We got a hazard ratio of 1.03, with 95\% CI (1.013 - 1.049). In context:
	\\ ~\ 
	
		\textcolor{blue}{We estimate the hazard ratio of death comparing groups one year apart in age with the same disease stage is 1.03, with the older group having the higher hazard. The data would not be surprising if the true hazard ratio were between 1.013 and 1.049. We reject the null hypothesis of a hazard ratio of 1 against the alternative that the hazard ratio is not 1 (p $<$ 0.001). There is significant evidence of an association between age and the instantaneous risk of death.}
\end{frame}

\begin{frame}{Cox proportional hazards regression: effect modification}
	Suppose we have a predictor of interest $X$ and binary potential effect modifier $Z$. We fit the proportional hazards model
	\begin{align*}
		h(t \mid X, Z) = h_0(t)e^{\beta_1X + \beta_2Z + \beta_3XZ}
	\end{align*}\pause 
	The hazard ratio comparing groups differing by 1 unit of $X$, for $Z = 0$, is
	\begin{small}
	\begin{align*}
		\frac{h(t \mid X = x+ 1, Z = 0)}{h(t \mid X = x, Z = 0)} = \frac{h_0(t)e^{\beta_1 (x + 1) + 0 + 0}}{h_0(t)e^{\beta_1x + 0 + 0}} = \frac{e^{\beta_1(x + 1)}}{e^{\beta_1 x}} = e^{\beta_1}
	\end{align*}
	\end{small}\pause 
	The hazard ratio comparing groups differing by 1 unit of $X$, for $Z = 1$, is
	\begin{small}
	\begin{align*}
		\frac{h(t \mid X = x+ 1, Z = 1)}{h(t \mid X = x, Z = 1)} = \frac{h_0(t)e^{\beta_1 (x + 1) + \beta_2 + \beta_3(x + 1)}}{h_0(t)e^{\beta_1x + \beta_2 + \beta_3x}} = \frac{e^{(\beta_1 + \beta_3)(x + 1) }}{e^{(\beta_1  + \beta_3)x}} = e^{\beta_1 + \beta_3}
	\end{align*}
	\end{small}\pause 
	The ratio of these two hazard ratios is $e^{\beta_3}$. 
\end{frame}

\begin{frame}{Cox proportional hazards regression: effect modification example}
	Does binarized \texttt{sex} (0 = male, 1 = female) modify the association between age and survival?
	\begin{align*}
		h(t \mid \texttt{age}, \texttt{sex}) = h_0(t)e^{\beta_1\texttt{age} + \beta_2\texttt{sex} + \beta_3\texttt{sex}\times\texttt{age}}
	\end{align*}
			\begin{center}
		\includegraphics[width = \textwidth]{figs/interaction.png}
	\end{center} 
\end{frame}

\begin{frame}{Cox proportional hazards regression: effect modification example}
	Our estimated (exponentiated) coefficients for \texttt{age} and \texttt{age:sex} were 1.054 and 0.981, respectively.
	\\ ~\ 
	
	Among males (\texttt{sex == 0}), we have a hazard ratio of 1.054, and among females (\texttt{sex == 1}), we have a hazard ratio of $1.054 \times 0.981 = 1.034$.  
	\\ ~\ 
	
	\textcolor{blue}{Among males, we estimate the hazard ratio comparing groups differing by one year in age to be 1.054, with older groups having the higher risk of death. Among females, we estimate the hazard ratio comparing groups differing by one year in age to be 1.034, with older groups having the higher risk of death. This difference in association would not be surprising if the true difference were between 0.941 and 1.024. We do not find significant evidence of effect modification by sex (p $= 0.38$).}
\end{frame}

\begin{frame}
\frametitle{Cox proportional hazards regression: assumptions}
\vspace{-0.5cm}
Aside from requiring independent censoring, the proportional hazards model makes another strong assumption; specifically, that we have
\begin{align*}
\frac{h(t \mid X = x + 1)}{h(t \mid X = x)}=  e^{\beta};
\end{align*}
\textit{For each $t$}, the hazard ratio is the same (and is $e^{\beta}$).
\\ ~\ 

This assumption is violated, for example, if the survival curves cross! 
\begin{center}
	\includegraphics[width = 0.7\textwidth]{figs/crossing_hazards.png}
\end{center}
\end{frame}

\begin{frame}{Cox proportional hazards regression: assumptions}
	If the exposure has a time-varying effect on the outcome, this is unlikely to hold. 
	\\ ~\ 
	
	For example, if you are studying the risk of COVID-19 infection comparing those who have recently been infected to those who have not, you would not expect proportional hazards to hold. 
	\begin{itemize}
		\item In the immediate aftermath of an infection, someone is very unlikely to be infected compared to someone who's immunologically naive. 
		\item Over time, immunity wanes, and reinfection becomes more likely. 
	\end{itemize}

	There are ways to assess proportional hazards using generalizations of the residual-vs.-fitted plots we saw in Chapter 1, although we won't get into them. For this course, just be aware of how strong this assumption may be.
\end{frame}


%\section{Summary}
%\begin{frame}
%\frametitle{SECTION 4: SUMMARY}
%Consider the inflammatory biomarkers data; here, normal CRP is $\leq 10$ mg/L, while abnormal CRP is $> 10$ mg/L.
%\begin{center}
%\includegraphics[height = 0.6\textheight]{figs/cum_haz_inflamm_crp.png}
%\end{center} \vspace{-0.6cm}
%First question: is the censoring here informative or uninformative?
%\end{frame}
%
%\begin{frame}
%\frametitle{Survival analysis: CRP}
%\textbf{Scientific question:} Is there \textcolor{orange}{a difference} in \textcolor{blue}{survival} comparing those with normal CRP to those with abnormal CRP?
%
%\textbf{Statistical question:} \vspace{-0.3cm}
%\begin{itemize}
%\item How do we want to quantify the outcome?
%\item[] 
%\begin{itemize}
%\item[]
%\item[]
%\end{itemize}
%\item How will we quantify association?
%\item[] 
%\begin{itemize}
%\item[]
%\item[] 
%\end{itemize}
%\item Are there other variables we need to adjust for?
%\end{itemize}
%\end{frame}
%
%\begin{frame}[noframenumbering]
%\frametitle{Survival analysis: CRP}
%\textbf{Scientific question:} Is there \textcolor{orange}{a difference} in \textcolor{blue}{survival} comparing those with normal CRP to those with abnormal CRP?
%
%\textbf{Statistical question:} \vspace{-0.3cm}
%\begin{itemize}
%\item How do we want to quantify the outcome?
%\item[] \textcolor{blue}{Two options:}
%\begin{itemize}
%\item \textcolor{blue}{Binary variable: survival until prior to first censoring time}
%\item \textcolor{blue}{Time-to-event variable}
%\end{itemize}
%\item How will we quantify association?
%\item[] 
%\begin{itemize}
%\item[]
%\item[] 
%\end{itemize}
%\item Are there other variables we need to adjust for?
%\end{itemize}
%\end{frame}
%
%\begin{frame}[noframenumbering]
%\frametitle{Survival analysis: CRP}
%\textbf{Scientific question:} Is there \textcolor{orange}{a difference} in \textcolor{blue}{survival} comparing those with normal CRP to those with abnormal CRP?
%
%\textbf{Statistical question:} \vspace{-0.3cm}
%\begin{itemize}
%\item How do we want to quantify the outcome?
%\item[] \textcolor{blue}{Two options:}
%\begin{itemize}
%\item \textcolor{blue}{Binary variable: survival until prior to first censoring time}
%\item \textcolor{blue}{Time-to-event variable}
%\end{itemize}
%\item How will we quantify association?
%\item[] \textcolor{orange}{Options (depending on study design):}
%\begin{itemize}
%\item \textcolor{orange}{Binary variable: linear/logistic/Poisson regression}
%\item \textcolor{orange}{Time-to-event variable: difference in survival curves, or hazard ratio}
%\end{itemize}
%\item Are there other variables we need to adjust for?
%\end{itemize}
%\end{frame}
%
%\begin{frame}
%\frametitle{Survival analysis: CRP}
%Survival analysis options:
%\begin{itemize}
%\item Are the \textcolor{blue}{survival curves}  \textcolor{orange}{different}, comparing those with normal CRP to high CRP, with the same \textcolor{orange}{age, diabetes status, sex, BMI, and smoking status?}
%\item Is the \textcolor{orange}{hazard ratio} of \textcolor{blue}{death} different from 1, comparing those with normal CRP to high CRP, with the same \textcolor{orange}{age, diabetes status, sex, BMI, and smoking status}?
%\end{itemize}
%\end{frame}
%
%% KM curves
%\begin{frame}
%\frametitle{Survival analysis: Kaplan-Meier and logrank}
%\hspace*{0.5cm}\includegraphics[width=0.9\textwidth]{figs/inflamm_km_crp.png}
%\end{frame}
%
%% unadjusted log rank
%\begin{frame}
%\frametitle{Survival analysis: Kaplan-Meier and logrank}
%\includegraphics[width = 1\textwidth]{figs/inflamm_logrank_crp.png}
%\end{frame}
%
%% adjusted log rank
%\begin{frame}
%\frametitle{Survival analysis: Kaplan-Meier and logrank}
%\includegraphics[width = 1\textwidth]{figs/inflamm_logrank_crp_adjust_1.png}
%\end{frame}
%
%\begin{frame}
%\frametitle{Survival analysis: Kaplan-Meier and logrank}
%\includegraphics[width = 1\textwidth]{figs/inflamm_logrank_crp_adjust_3.png}
%\end{frame}
%
%\begin{frame}
%\frametitle{Survival analysis: Kaplan-Meier and logrank}
%
%\textcolor{green}{Based on a logrank test}, \textcolor{blue}{these data provide evidence to suggest that the survival function in those with normal CRP is statistically significantly different from the survival function in those with abnormal CRP at the 0.05 level (p $< 0.001$)}. \textcolor{orange}{We conclude that abnormal CRP is associated with decreased survival compared with normal CRP}.
%
%\textcolor{green}{Based on a logrank test}, \textcolor{blue}{these data provide evidence to suggest that the survival function in those with normal CRP is statistically significantly different from the survival function in those with abnormal CRP, among those with the same age, diabetes status, sex, and BMI, at the 0.05 level (p $< 0.001$)}. \textcolor{orange}{We conclude that abnormal CRP is associated with decreased survival compared with normal CRP, adjusting for age, diabetes status, sex, and BMI}.
%\end{frame}
%
%% PH output
%
%\begin{frame}
%\frametitle{Survival analysis: proportional hazards regression}
%Regression model: 
%\begin{align*}
%\log & \{h(t \mid \text{CRP $>$ 10}, \text{Age}, \text{Diabetes}, \text{Male}, \text{BMI}\} =  \log\{h_0(t)\} \\
%&+ \beta_1 (\text{CRP $>$ 10}) + \beta_2 \text{Age} + \beta_3 \text{Diabetes} + \beta_4 \text{Male} + \beta_5 \text{BMI}
%\end{align*}
%\hspace*{0.5cm}\includegraphics[width=0.9\textwidth]{figs/inflamm_hazard_crp_1.png}
%\end{frame}
%
%\begin{frame}
%\frametitle{Survival analysis: proportional hazards regression}
%\hspace*{0.5cm}\includegraphics[width=0.9\textwidth]{figs/inflamm_hazard_crp_2.png}
%\end{frame}
%
%% PH reporting results
%\begin{frame}
%\frametitle{Survival analysis: proportional hazards regression}
%Reporting results: 
%\begin{enumerate}
%\item Identify regression coefficient(s) of interest
%\item Report an estimate, and interpret
%\item Report a confidence interval, and interpret
%\item Report a p-value, and interpret
%\item Add a conclusion tying back to the scientific question
%\end{enumerate}
%\end{frame}
%
%\begin{frame}
%\frametitle{Survival analysis: proportional hazards regression}
%\begin{align*}
%\log & \{h(t \mid \text{CRP $>$ 10}, \text{Age}, \text{Diabetes}, \text{Male}, \text{BMI}\} =  \log\{h_0(t)\} \\
%&+ \beta_1 (\text{CRP $>$ 10}) + \beta_2 \text{Age} + \beta_3 \text{Diabetes} + \beta_4 \text{Male} + \beta_5 \text{BMI}
%\end{align*}
%
%Reporting results: 
%\begin{enumerate}
%\item Identify regression coefficient(s) of interest
%\item Report an estimate, and interpret
%\item Report a confidence interval, and interpret
%\item Report a p-value, and interpret
%\item Add a conclusion tying back to the scientific question
%\end{enumerate}
%\end{frame}
%
%\begin{frame}[noframenumbering]
%\frametitle{Survival analysis: proportional hazards regression}
%\begin{align*}
%\log & \{h(t \mid \text{CRP $>$ 10}, \text{Age}, \text{Diabetes}, \text{Male}, \text{BMI}\} =  \log\{h_0(t)\} \\
%&+ \beta_1 (\text{CRP $>$ 10}) + \beta_2 \text{Age} + \beta_3 \text{Diabetes} + \beta_4 \text{Male} + \beta_5 \text{BMI}
%\end{align*}
%
%Reporting results: 
%\begin{enumerate}
%\item Identify regression coefficient(s) of interest: $\textcolor{blue}{e^{\beta_1}}$
%\item Report an estimate, and interpret
%\item Report a confidence interval, and interpret
%\item Report a p-value, and interpret
%\item Add a conclusion tying back to the scientific question
%\end{enumerate}
%\end{frame}
%
%\begin{frame}
%\frametitle{Survival analysis: proportional hazards regression}
%
%\textit{2. Report an estimate, and interpret.}
%
%Based on proportional hazards regression, we estimate that the \textcolor{orange}{ratio of the instantaneous risk of death}, comparing those with abnormal CRP to those with normal CRP but with the same age, diabetes status, sex, and BMI, is \textcolor{orange}{1.68}, where those with abnormal CRP have a higher instantaneous risk of death. \pause
%
%\textbf{Option 2:} Based on proportional hazards regression, we estimate that the \textcolor{orange}{instantaneous risk of death is 1.68 times higher} among those with abnormal CRP compared to those with normal CRP... \pause
%
%\textbf{Option 3:} Based on proportional hazards regression, we estimate that the \textcolor{orange}{instantaneous risk of death is 68\% times higher} among those with abnormal CRP compared to those with normal CRP... 
%\end{frame}
%
%\begin{frame}
%\frametitle{Survival analysis: proportional hazards regression}
%
%\textit{3. Report a confidence interval, and interpret.}
%
%Based on a 95\% confidence interval, this estimate would not be judged unusual if the true hazard ratio were between 1.36 and 2.08.
%
%\textit{4. Report a p-value, and interpret.}
%
%These data provide evidence to suggest that this hazard ratio is statistically significantly different from one (p $< 0.001$).
%
%\textit{5. Add a conclusion tying back to the scientific question.}
%
%Based on these results, we have evidence to suggest that the instantaneous risk of death is larger in those with abnormal CRP, after adjusting for age, diabetes status, sex, and BMI.
%\end{frame}

%\begin{frame}
%\frametitle{Survival analysis: assumptions}
%
%Dealing with survival data is complicated, but \textcolor{blue}{we may gain information over simply dichotomizing a time-to-event variable}, and \textcolor{cyan}{sometimes our scientific question does involve the time}.
%
%All of the techniques we have learned rely on the assumption of \textcolor{blue}{uninformative censoring}: those people who are censored are not systematically different from the people in your target population.
%
%This is \textcolor{red}{difficult to assess}, unless you know some additional information about the participants. However, if you are following patients, you may be able to tell based on trends if the patient leaves due to their outcome getting worse.
%
%\end{frame}
%
%\begin{frame}
%\frametitle{Survival analysis: assumptions}
%
%The Kaplan-Meier estimator of the survival function and the logrank test only rely on uninformative censoring -- however, these become difficult to interpret when adjusting for covariates.
%
%The proportional hazards regression model is a convenient alternative, but makes a potentially strong assumption on the hazard ratio between groups defined by the predictor of interest -- namely, that the hazard is proportional at each time $t$.
%
%Despite this limitation, proportional hazards regression is widely used in survival analysis, \textcolor{red}{but the results are often misinterpreted}.
%\end{frame}

\end{document}